% NOTE - this is only a template without real arguments
\begin{entry}{Initial Log API Functionality}{Mar 18, 2021}
    \objective 
    
    Get some logging done
    
    \outline
    
    \begin{enumerate}
        \item Meet with Shiva
        \item Write a script that subscribes to the simulation output topic
        \item Put this data somewhere
    \end{enumerate}
    
    \procedures
    
    Today's task was to take what I've learned so far and begin to apply it in unique ways. The first thing I did was
    write (or more accurately adapt) a callback function. This is required whenever you subscribe to a topic. In the
    case of the simulation output, the python thing calls this function once every three seconds. So the ideal thing
    to do was, after a lot of testing, use this callback to write a line of data to a dataframe. In this early step
    I skipped the middleman and just wrote to a csv. This will allow me to easily test and troubleshoot the eventual
    data going into the pandas dataframe.

    I went of script and tried subscribing to the simulation log script. This is useful because it gives more data in
    the python terminal about the simulation and, importantly, the status of the simulation. This can be used to stop
    the script from running infinitely.


    \results
    
    Logs are being generated containing who-knows-what, I just used a random message for the topic. I think it's
    pretty much the entire simulation output. Next step is to filter it down to more specific data and put that in a
    pandas dataframe with useful labels and, importantly, timecodes. Then, sadly, I need to rewrite the entire script
    since it's basically a scratchpad file at the moment.
    
\end{entry}